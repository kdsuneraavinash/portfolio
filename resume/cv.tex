\documentclass{cv}

\usepackage[left=0.6in,top=0.5in,right=0.6in,bottom=0.5in]{geometry}    % Document margins
\newcommand{\tab}[1]{\hspace{.2667\textwidth}\rlap{#1}}
\newcommand{\itab}[1]{\hspace{0em}\rlap{#1}}

\imagename{face.png}
\name{K. D. Sunera Avinash Chandrasiri}                                 % Name
\subtitle{344/1, Moonamalgahawatta, Duwa Temple Road, Kalutara South.}  % Address
\subtitle{(076) 833 6850 \\ \href{mailto:suneraavinash.17@cse.mrt.ac.lk}{suneraavinash.17@cse.mrt.ac.lk}} 
\header{true}
% Phone number, email

\begin{document}
\vspace{5pt}

%----------------------------------------------------------------------------------------
%	EDUCATION SECTION
%----------------------------------------------------------------------------------------

\begin{rSection}{Profile}
A third-year computer science undergraduate who has a passion for algorithms and problem-solving.  
Also interested in web technologies and mobile app development. A GNU/Linux Enthusiast.\\
    Github profile: \url{https://github.com/kdsuneraavinash} \\
    LinkedIn profile: \url{https://www.linkedin.com/in/kdsuneraavinash} \\
    Hackerrank profile: \url{https://www.hackerrank.com/kdsuneraavinash} \\ 
    Portfolio: \url{https://kdsuneraavinash.github.io/portfolio} 
\end{rSection}

\begin{rSection}{Education}

{\bf University of Moratuwa}                                \hfill {\em September 2017 - Present} 
\\ Undergraduate, Bsc (Computer Science and Engineering)    \hfill { Current SGPA: 4.08/4.2 }
\\ Dean's List - Semester 1, 2, 3 \\
\\{\bf Kalutara Vidyalaya – National School}                \hfill {\em July 2013 - June 2017} 
\\ GCE Advanced Level Examination                           \hfill { Z-Score: 2.680 }
\\ All A Passes in Physical Sciences stream (Island rank 76\ss{th}, District Rank 2\ss{nd}) 
\end{rSection}

%----------------------------------------------------------------------------------------
%	TECHNICAL STRENGTHS SECTION
%----------------------------------------------------------------------------------------

\begin{tSection}{Technical Strengths}{
Languages (Strong Familiarity)      \ & Python, Dart, Java, JavaScript \\
Languages (Medium Familiarity)      \ & C++, TypeScript, Golang, PHP \\
Web Development                     \ & NodeJS, React, HTML5, CSS, Bootstrap, PHP, WordPress \\
Databases                           \ & MySql, PostgreSql, Firebase Firestore \\
}\end{tSection}

%----------------------------------------------------------------------------------------
%	COMPETITION AWARDS
%----------------------------------------------------------------------------------------

\begin{rSection}{Competition Awards}{
{\bf IEEExtreme 12.0 \& 13.0 (IEEE) | Top 100 (global)}                         \hfill {\em October 2018 \& 2019} 
\\A 24-hour international competitive programming competition organized by IEEE and attended by more than 4000 teams.\\ \\
{\bf ACES Coders v8.0 (University of Peradeniya) | Winners}                     \hfill {\em February 2020} 
\\A 12-hour competitive programming competition organized by the Association of Computer Engineering Students (ACES) of the Department of Computer Engineering, Faculty of Engineering, the University of Peradeniya with the participation of more than 100 teams from about 25 institutions worldwide.\\ \\
{\bf UOJ Coders v1.0 (University of Jaffna) | Winners}                          \hfill {\em March 2019} 
\\A 12-hour competitive programming competition organized by University of Jaffna.\\ \\
{\bf HackX 2019 (University of Kelaniya) | Winners}                             \hfill {\em September 2019} 
\\An innovative startup challenge conducted by the Management and IT faculty of the University of Kelaniya in which our team designed and developed an Android/iOS mobile app named `Teleport' to facilitate ride-sharing for parcel delivery.\\ \\
{\bf IESL RoboGames '18 (IESL \& University of Moratuwa) | Winners}             \hfill {\em October 2018} 
\\A computer-vision based robotics competition conducted as part of the Techno exhibition `18 in which the task was to build an autonomous quadruped robot that could navigate a maze by using drawn arrows as guidelines.\\ \\
{\bf 14\ss{th} YCS Competition (Ministry of Education \& FITIS) | Winner}       \hfill {\em August 2015} 
\\An island-wide software development competition organized by the Ministry of Education in which my solution was a learning management system designed to teach students who have negligible knowledge about mathematics.\\ \\
{\bf XOBot `19 (IESL \& University of Jaffna) | Runners Up}                     \hfill {\em October 2019} 
\\A robotic manipulator design competition conducted as part of the Techno exhibition `19 in which the task was to design a Tic-Tac-Toe playing robotic arm.\\ \\
{\bf Reakhack 2.0 (University of Kelaniya) | Runners Up}                        \hfill {\em November 2019} 
\\A 24-hour hackathon organized by the Software Engineering Students’ Association of the University of Kelaniya in which the task was to build a real-time Q\&A site without using third-party services.\\ \\
{\bf IDEA Challenge (University of Moratuwa) | Runners Up}                      \hfill {\em October 2018} 
\\Mobile app development competition organized as part of the Techno exhibition '18 in which our team built an Android app that aims to make awareness of Dengue and take measures to prevent its occurrence and spread.\\ \\
{\bf Sri Lanka Robotics Challenge `17 (University of Moratuwa) | Second Runners Up} \hfill {\em January 2018} 
\\A robotics competition organized by the E-Club of the University of Moratuwa.\\ \\
{\bf Hash Code 2019 (Google) | Sri Lankan First Place \& 260\ss{th} globally}   \hfill {\em February 2019} 
\\A team programming competition, organized by Google, for students and professionals around the world in which the teams are given an engineering problem to solve.
}\end{rSection}

%--------------------------------------------------------------------------------
%    Honours and Awards (Other)
%--------------------------------------------------------------------------------

\begin{rSection}{Honours \& Awards (Other)}
{\bf Hiran Chathura Kulasekara Throphy}                                                         \hfill {\em 2016} 
\\Award for best performing student in Physical Science Stream, Kalutara Vidyalaya - National School.\\ \\
{\bf National Level School Software Championship (Ministry of Education) | Merit}               \hfill {\em 2015}\\ 
{\bf ICT Competition - Grade 11 (Ministry of Education) | Western Province 1\ss{st} place}      \hfill {\em 2013}\\ 
{\bf Science Competition - Grade 8 (Ministry of Education) | Western Province 2\ss{nd} place}   \hfill {\em 2010}\\ 
{\bf Social Science Competition - Grade 8 (Ministry of Education) | All Island 5\ss{th} place}  \hfill {\em 2010} 
\end{rSection}

%--------------------------------------------------------------------------------
%    Extra-Curricular
%--------------------------------------------------------------------------------

\begin{rSection}{Extra-Curricular}
{\bf Teaching Mathematics to students in rural areas | Soyuru Sathkaraya}       \hfill {\em Sept 2017 - Nov 2017}
\\"Soyuru Sathkaraya" is an annual volunteer program organized by the Students' Union of University of Moratuwa to improve the mathematics knowledge and abilities of the students in rural schools in Sri Lanka.
\end{rSection}

%--------------------------------------------------------------------------------
%    Projects
%--------------------------------------------------------------------------------

\begin{rSection}{Projects}
{\bf Theme Provider | Open source plugin}                        \hfill {\em June 2019} 
\\An easy to use theme controlling plugin for Flutter which automatically rebuilds UI on theme changes.\\
\url{https://pub.dev/packages/theme\_provider} \\
\\{\bf Rise of the Pharaohs Scavenger hunt app}                 \hfill {\em October 2019 - February 2020} 
\\Scavenger-hunt mobile app created as the invitation for the CSE event – Rise of the Pharaohs.\\
\url{https://github.com/kdsuneraavinash/cse-night-app}. \\
\\{\bf Teleport Mobile App(Android/iOS)}                        \hfill {\em August 2019 - September 2019} 
\\App developed for HackX competition in order to use sharing economy in transferring goods. \\
\\{\bf Tic-Tac-Toe Playing Robot Arm | Mobile app integrated robot}     \hfill {\em October 2019}
\\This was a vision-based approach to developing a tic-tac-toe playing robot using Arduino, Android and OpenCV.\\
\\{\bf Open Inventory System | Inventory Management System}     \hfill {\em February 2020 - June 2020} 
\\Automated inventory management system for computer labs with RBAC. The system consisted of a mobile application and a web application.\\
\url{https://github.com/openinventoryorg}.\\
\\{\bf Neomerce | Single Vendor e-commerce application}         \hfill {\em November 2019 - December 2019}
\\Single vendor e-commerce web application prototype made for Semester 4 CS3042 Database Systems Module.
Allows buying products and report generation.\\
\url{https://github.com/kdsuneraavinash/neomerce}\\
\\{\bf Piano Tiles Bot | Open source project}                   \hfill {\em November 2018}
\\A hobby project I did which captures a window from an emulator and uses openCV to process the image and press according keys. This bot was intended to play piano tiles game and can reach 4000+ scores easily.\\
\url{https://github.com/kdsuneraavinash/python-projects/tree/master/piano-tile-bot}\\
\url{https://youtu.be/e-hemLmjIWc}\\
\end{rSection}

%--------------------------------------------------------------------------------
%    Non-related Refrees
%--------------------------------------------------------------------------------

% \begin{rSection}{Non-Related Refrees}
% \end{rSection}


\end{document}

